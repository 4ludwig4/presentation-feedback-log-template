\documentclass{scrartcl}
\usepackage[english]{babel}
\usepackage{blindtext}
\usepackage{geometry} 	% Page Layout
\usepackage{xcolor}		% Response coloring 
\usepackage{scrextend}	% Indent text blocks 


% Define environments 
\newcounter{feedbackcounter}[section]
\newenvironment{feedback}{\refstepcounter{feedbackcounter}\par\medskip\noindent \textbf{\thefeedbackcounter. Feedback:} \rmfamily}{\medskip}
\newenvironment{response}{\par\medskip\noindent \begingroup\color{blue}\begin{addmargin}[3em]{0em}\textbf{Response: }\ignorespaces }{\end{addmargin}\endgroup\medskip \ \\}

% Set page layout 
\geometry{a4paper, top=15mm, left=20mm, right=20mm, bottom=20mm,
headsep=1mm, footskip=12mm}

%%%%%%%%%%%%%%%%%%%%%%%%%%%%%%%%%%%%%%%%%%%%%%%%%%%%%%%%%%%%%%%%
% Metadata
\def\doctype{Master's Thesis} % FIXME
\def\doctitle{Fancy Thesis Title} % FIXME
\def\supervisor{Prof. Dr. Stephan Krusche} % FIXME
\def\advisor{Prof. Dr. Stephan Krusche} % FIXME
\def\author{Name Name} % FIXME
\def\presentation_date{04.05.22} % FIXME
%%%%%%%%%%%%%%%%%%%%%%%%%%%%%%%%%%%%%%%%%%%%%%%%%%%%%%%%%%%%%%%%


% Define title from metadata
\title{Presentation Feedbacklog \\ \doctype \ \author \ \\ \doctitle}


\begin{document}
	\maketitle
	\noindent
	\rule{\linewidth}{.1mm}	
	\begin{center}
		\begin{tabular}{ll}
			Author:            &  \author     \\
			Supervisor: 	       &  \supervisor \\				
			Advisor:	           &  \advisor    \\
			Presentation date: &  \presentation_date
		\end{tabular}
	\end{center}
	\noindent
	\rule{\linewidth}{.1mm}	
	
	%%%%%%%%%%%%%%%%%%%%%%%%%%%%%%%%%%%%%%%%%%%%%%%%%%%%%%%%%%%%%%%%
	% Start adding your feedback here
	%%%%%%%%%%%%%%%%%%%%%%%%%%%%%%%%%%%%%%%%%%%%%%%%%%%%%%%%%%%%%%%%
	\begin{feedback} % Example Feedback
		Since the attention span of the audience is rather short, try getting to the problem as fast as possible. Several slides in the presentation display redundant or repetitive information and could be deleted
	\end{feedback} % Example Rresponse	
	\begin{response}
	        To  get  to  the  problem  earlier  in  the  presentation  as  well  as  to  present  only relevant  information,  some  of  the  slides  are  removed  (Agenda)  and  some  of  them  are  merged (Context, Motivation and Objectives), therefore the “Problem” slide is now at slide 4. 
	\end{response}
	
	
	\begin{feedback} % Example Feedback
		The structure of the presentation is rather unique; it would be great if you could help the audience to follow it easier.
	\end{feedback}		
	\begin{response} % Example Rresponse
		To help the audience to follow the presentation easier, a progress-bar was added at the bottom of the slides to show the milestones of the presentation and highlight the one that indicates the current slide.
	\end{response}
	
	
\end{document}
